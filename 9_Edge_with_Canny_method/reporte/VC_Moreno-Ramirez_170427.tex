\documentclass[conference]{IEEEtran}
\usepackage[spanish]{babel}
\usepackage[latin1]{inputenc}
\usepackage{blindtext, graphicx}
\usepackage{subfigure}
\usepackage{mdwmath}
\usepackage{mdwtab}
\usepackage{subfig}
\usepackage{amsmath}

\begin{document}
\title{ Detecci\'on de bordes utilizando el m\'etodo de Canny }
\author{\IEEEauthorblockN{Walter Alejandro Moreno Ram\'irez}
\IEEEauthorblockA{Departamento de Estudios Multidisciplinarios\\
Universidad de Guanajuato\\
Yuriria, Guanajuato\\
Correo: wa.morenoramirez@ugto.mx}}

\maketitle
\renewcommand\abstractname{Abstract}
\begin{abstract}
This article describes what are the edges of an image, as they can be detected by obtaining the gradient of a region of the image. Its advantages and applications will also be stated. \\\\
\end{abstract}

\begin{IEEEkeywords}
Pixel, p\'ixeles, convoluci\'on, canny, smoothing, suavizado, gradiente, segmentaci\'on ,primera derivada, funci\'on, C++, OpenCV, ventana, m\'ascara, vecindario.
\end{IEEEkeywords}

\IEEEpeerreviewmaketitle
\section{Introducci\'on} 


\section{Metodolog\'ia}


\section{Resultados}


\section{Conclusiones}


%$\begin{bmatrix}
% 1 & 1 & 1 & 1 \\
% 1 & 1 & 1 & 1 \\
% 1 & 1 & 1 & 1 \\
% 1 & 1 & 1 & 1 \\
%\end{bmatrix}$

%\begin{thebibliography}{1}
%    \bibitem{IEEEhowto:kopka}
%    H.~Kopka and P.~W. Daly, \emph{A Guide to \LaTeX}, 3rd~ed.\hskip 1em plus
%      0.5em minus 0.4em\relax Harlow, England: Addison-Wesley, 1999.
%\end{thebibliography}

\end{document}
